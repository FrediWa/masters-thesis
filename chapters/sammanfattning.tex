\subsection*{Inledning}
\addcontentsline{toc}{subsection}{Inledning}
En person med absolut gehör kan stämma en gitarr med att helt enkelt lyssna på dess ljud och vrida på stämskruven tills ljudet låter rätt. En person som inte har absolut gehör måste lysna på en referens, t.ex. en stämgaffel, och jämföra ljuden. Hen hör om gitarrens ton är lägre eller högre och kan justera ljudet i rätt riktning tills gitarren och referensen är samma. Båda fallen är exempel på tonhöjdsidentifiering. Detta är något datorer kan hjälpa med och därmed har man studerat och utvecklat flera olika metoder. Tonhöjd har en stark anknytning till oskillationsfrekvensen på en ljudsignal. Vissa metoder fokuserar direkt på signalen medan andra omvandlar signalen till något som är lättare att jobba med.

Detta arbete fokuserar på Fourier transformen och de signalbehandlingsmetoder som krävs för tonhöjdsidentifiering. Systemet ska fungera i realtid och snurra i webbläsarmiljö så att det lätt kan integreras i webbapplikationer.

\subsection*{Fouriernalys}
\addcontentsline{toc}{subsection}{Fourieranalys}
Joseph Fourier jobbade på en differentialekvation och tänkte att det skulle vara lättare att integera och derivera funktioner om de vore sinus- eller cosinusvågor. Han började utforska tanken att representera funktioner som sinus och cosinus vågor och till slut utvecklade han det vi kallar i dag för Fourierserier. En generisk Fourier serie (på intervallet $[0, 2\pi]$) ser ut på följande vis

$$f(t) = \frac{a_0}{2} + \sum_{n=1}^{\infty}a_ncos(nt)+\sum_{n=1}^{\infty}b_nsin(nt)$$

Med detta menas att vilken som helst periodisk funktion på intervallet kan uttryckas som en kombination av olika sinus- och cosinus vågor. 

Fouriertransformen, till skillnad från Fourierseries, kan användas för icke-periodiska funktioner. Fouriertransformen definieras ofta på följande vis

$$\hat{f}(x) = \int_{-\infty}^{\infty} f(t)e^{-i2\pi x t} dt$$

Fouriertransformen av en signal är en ny signal i det så kallade frekvensdomänet. Frekvensdomän signalen beskriver vilka komponenter signalen innehåller och hurdana de är. Alternativt kan man se på Fouriertransformen som en funktion som mäter korrelationen mellan en funktion och alla sinus- och cosinusvågor. Om inte en viss sinusvåg korrelerar med funktionen
är den inte en del av funktionen. Korrelationen fungerar på grund av icke-ortogonalitet, alltså att integrering av $f(x)sin(nx)$ är $\pi$. Om funktion inte innehåller en viss komponent, kommer integreringen ge $0$ för korreleringen mellan funktionen och komponenten.

Fourier transformen har en diskret variant som ofta kallas för DFT. 
$$X_k = \sum_{n=0}^{N-1} x_ne^{-\frac{i2\pi kn}{N}}$$
Denna funktion gör i princip samma som den kontinuerliga transformen men används för diskret data.

I frekvensdomänet kan man lätt modifiera signalen. T.ex. filtrering av höga ljud kunde man lätt göra med att helt enkelt radera komponenterna. Frekvensdomänet kan också användas för att lagra information för trådlös kommunikation. Man kan dock varken spela upp eller sända signalen i frekvensdomänet. Man skulle måste omvandla signalen tillbaka till tidsdomänet. Detta gör man med inverstransformen. 

$$ f(t) = \frac{1}{\pi}\int_{-\infty}^{\infty} \hat{f}(\zeta)e^{i2\pi\zeta t} d\zeta$$
och med den diskreta inverstransformen
$$ x_n = \frac{1}{N}\sum_{k=0}^{N-1} X(k)e^{\frac{i2\pi kn}{N}}$$

När man gör beräkningar med en dator används ofta en algoritm som kallas för Fast Fourier transform (FFT). Den kallas snabb för att den producerar samma resultat men  gör det mer effektivt. Effektiviten är så hög att den verkligen är snabbare, speciellt för stora data mängder. FFT-algoritmen noteras vara ca 12800 gånger snabbare än DFT algoritmen för 512 000 data punkter.  
\subsection*{Tonhöjdsidentifiering}
\addcontentsline{toc}{subsection}{Tonhöjdsidentifiering}
Tonhöjdsidentifiering handlar om att räkna ut vilken tonhöjd en uppspelad not är. Metoder för att identifiera tonhöjd går in i en av tre kategorier; tidsdomän, frekvensdomän och statiskta/maskininlärnings metoder. En populär tids-domän metod är autocorrelation där signalen jämförs med en förskjuten version av sig själv. 

Ljud är en tryckskillnad i ett medium och modelleras ofta som en våg. T.ex. en högtalare som spelar tonen A4 trycker ut luft med en frekvens på 440Hz. En ren 440Hz signal kommer dock att låta rå och otrevlig. Ett vackert ljud så som ett A4 på ett riktigt instrument innehåller mycket annat än bara 440Hz. Detta kallas klangfärg eller timbre.

Tonhöjdens egentliga natur är inte alltid klart. Man kan säga att tonhöjden är hur högt eller lågt en lyssnare anser att ljudet är. Man kan även säga att tonhöjden är grundfrekvensen i ett ljud. Även där kan det uppstå ett problem där grundfrekvensen inte verkligen finns som en del av signalen. Detta kallas för \textit{missing fundamental} och fenomenet uppstår då harmonier bygger på varandra och bildar ett prominant ljud med en frekvens som är den största gemensamma delaren på harmoniernas frekvenser.

För tonhöjdsdetektering i frekvensdomänet börjar man med att omvandla signalen med hjälp av en Fouriertransform. Om man har ett längre ljud, kunde man använda en Short Time Fourier Transform (STFT) som kort sagt delar upp en signal i mindre fönster som kan analyseras skillt. I detta fall byggs en realtids detektor, vilket betyder att fönstren bildas naturligt med tiden, så en normal FFT används. I frekvensdomänet visas vilka komponenter finns i signalen och härifrån ska en av dem väljas.

För processering av frekvensdomänet finns det flera algoritmer. Harmonic Product Spectrum (HPS) och Constant False Alarm Rate (CFAR) diskuteras i detta arbete. HPS nersamplar frekvensdomänet och multiplicerar iterationerna vilket skapar ett nyt frekvens domän där harmonier förstärker den största gemensamma delaren. HPS-algoritmen är användbar för den hittar missing fundamental om den saknas. HPS-algoritmen däremot fungerar väldigt dåligt om det finns få harmonier och kan i sådana fall ge fel svar. CFAR är en algoritm som hittar toppar i frekvensdomänet med att skapa dynamisk tröskel med hjälp av att applicera statistiska metoder på signalen. Om en punkt är ovanför tröskeln är den antagligen inte brus. HPS fungerar bättre för tonhöjdsdetektering men CFAR kunde hjälpa med att implementera en dynamisk HPS där iterationer ökas eller minskas beroende på hur många toppar CFAR har hittat. 

Tonhöjdsdetektorn ska fungera i realtid. Vad en acceptabel latens är är kontextuellt. För renderering är det 33ms för att uppnå 30FPS. För realtids paketspårning är en uppdatering per timme kanske tillräckligt. Eftersom syftet är att användaren ska få feedback direkt är latensen mest beroende på användarens egen reaktions tid. Att minimera latensen är svårt eftersom Fouriertransformens frekvensresolution är inverst proportionell till mängden datapunkter som samlas. Man kan kringå detta med en metod som kallas \textit{zero-padding}, där man skapar artificiella datapunkter som inte har en inverkan på signalen. 

\subsection*{Implementering}
\addcontentsline{toc}{subsection}{Implementering}
Tonhöjdsdetektorn implementeras för webben och därmed används Web Audio APIn mycket. Denna API ger åtkom till användarens mikrofon och andra redskap som kan användas för audio processering. Om man får rådata från mikrofonen kan man enkelt processera det med bibliotek såsom FFT.js. Hur man får rådata från mikrofonen är dock inte enkelt. 

En lösning till att få rådatat är att använda sig av Web Audio APIns audio graf. Med APIn kan man bygga upp en kontrollgraf där noder skickar data till varandra. Noderna har olika roller; källnoder, modifierarnoder och destinationsnoder. Källnoden för detta ändamål är en nod som får data från en mikrofon och inga destinations noder används eftersom ljudet inte behövar spelas tillbaka till användaren. Inga av de existerande modifierar noderna kan användas för tonhöjdsdetektorn. En AnalyserNode kör Fourier transformen, men denna kan man inte zero-padda, vilket måste göras för att få tillräcklig frekvens resolution i realtid. 

Lösningen är att bygga en egen nod med hjälp av WorkletNodes. WorkletNodes tillåter användaren att bygga egen funktionalitet till grafen. Tanken är att man bygger en nod som helt enkelt samlar rådata och ger det över till en normal JavaScript funktion. Detta ger även flexibilitet att använda grafen vilket gör det enkelt att validera och testa. För att testa systemet mot en ljudfil gäller det bara att byta ut mikrofon noden mot en audiofil nod.  

Den implementerade tonhöjdsdetektorn är en pipeline där audio grafen samlar data, skickar det vidare till en analysfunktion som kör FFT, HPS, en del post processering och till sist presenterar användaren med en not. Allt detta händer på ca 120ms. 

\subsection*{Resultat}
\addcontentsline{toc}{subsection}{Resultat}
Tonhöjdsdetektorn testas med hjälp av ljudfiler som Finlands-svenska mansångarförbundet har försett. Tonhöjdsdetektorn analyserar alla fyra stämmor av Finlandia och resultatet jämförs med notbladet. 

Resultaten indikerar problem i HPS-algoritmen. Det vanligaste är att tonhöjdsdetektorn ger resultat som är fel med en oktav. Olika mängder iterationer för HPS testades även och slutsatsen är att en dynamisk HPS (en variant av HPS där antalet iterationer kan automatiskt justeras på basen av toppar i frekvensdomänet) vore bra då fyra iterationer presterar bäst i vissa fall och 6 iterations bäst i andra fall.

Genom att använda detektorn för att stämma en gitarr kan dess responsivitet även testas. Testet visar att detektorn är tillräckligt responsiv för att användaren ska kunna mer eller mindre kontinuerligt vända stämskruven. Däremot visar det sig att frekvensresolutionen är för hög för att stämma ett instrument precist. 

\subsection*{Slutsats}
\addcontentsline{toc}{subsection}{Slutsats}
I detta arbete utforskades Fouriertransformen och metoder för att analysera frekvensdomänet för tonhöjdsidentifiering. Syftet med arbetet var att utveckla en tonhöjdsdetektor som kan användas för att träna sång.

Resultatet av arbetet är en JavaScript klass med stödfunktioner som en användare kunde använda i sin egen applikation. Även om applikationen inte har någon API i detta skede, är den byggd så att det i framtiden ska vara lätt att ändra koden så att användaren inte behöver modifier källkoden för att få den funktionalitet hen vill. 

Även om tonhöjdsdetektorn ger rätt resultat största delen av tiden är det inte tillräckligt för att användningen av den ska vara pålitlig. Största delen av de fel detektorn gör är oktavfel.
