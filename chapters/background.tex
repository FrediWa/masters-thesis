\section{Introduction}
This paper bla bla bla

\subsection{Fourier series and transform}
In the beginning of the 1800s, Joseph Fourier was working on a physics problem called the heat equation, a certain partial differential equation. He had the idea of expressing the original function as a sum of sine and cosine functions as these would integrate and differentiate easily, making the differential equation easier to solve. He eventually developed and introduced the idea of Fourier series, a way of expressing a function as a sum of trigonometric functions. 

If a function has a period of $2\pi$, the Fourier series takes the form $$f(t) = \frac{a_0}{2} + \sum_{n=1}^{\infty}a_ncos(nt)+\sum_{n=1}^{\infty}b_nsin(nt)$$. Using this representation and a the properties of a few integrals of trigonometric functions, one can derive formulae for $a_0$, $a_n$ and $b_n$.

As an example, the following $$\pi -2sin(x) -sin(2x) -\frac{2}{3}sin(3x) -\frac{1}{2}sin(4x) ... $$ series is a sort of sawtooth wave.

The Fourier Transform, abbreviated as FT, was developed after Fourier's big idea. It takes a function in time-domain (a signal as a function of time, like a song) and outputs a function in frequence-domain, a kind of description for which kinds of sinusoids make up the original signal. 

Even though the the FT is an incredibly powerful function, it's formula is compact. 
$$\hat{f}(\zeta) = \int_{-\infty}^{\infty} f(t)e^{i2\pi-\zeta t} dt$$
The variable \zeta is used here to emphasize that the output is complex.
\subsection{Discrete fourier transform}
% What is it plainly
\subsubsection{Definition}
% Math formula
\subsubsection{DFT matrix}

\subsection{Fast fourier transform}
\subsubsection{FFT matrix}
\subsubsection{Algorithmic Complexity}

\subsection{Signal processing applications}

\subsection{Use in Digital audio processing}

