\begin{abstract}
Receiving correct feedback is vital for learning anything new. Learning to sing can be difficult for a variety of reasons, but knowing when one hits the right notes — and when one does not — is a good start. Computers have for a long time helped people with learning, be it through e-learning platforms, forums for connecting with experts, or lately, with large language models for the ability to almost converse with an expert at any moment. Computers have also been aiding in tuning instruments, such as guitars, for some time. This work explores frequency-domain methods and post-processing techniques for pitch estimation, to enable users to receive instant feedback on their singing. System architecture for such a pitch detector is then planned and outlined, and implemented using the Web Audio API for use on the web. The pitch detector is tested using audio from real choir singers for a reliable reference. The resulting pitch detector lays a good foundation by being accurate, but falls short in certain aspects.  
\end{abstract}

\vspace{2cm}

\noindent \textbf{Keywords:} Pitch Detection, Audio Processing, FFT, Web Audio API, Harmonic Product Spectrum

\newpage 