The purpose of this work was to explore frequency-domain approaches to pitch detection for use in real time. It was constrained to focus on giving feedback specifically for male (TTBB) singers. Using the theory created by Joseph Fourier in conjunction with more modern signal processing techniques such as the FFT and Harmonic Product Spectrum, the system can process an incoming stream to continuously estimate the fundamental frequency. A sampler is used to convert the continuous estimate stream to a sequence of notes which can be compared offline with Dynamic Time Warping or in real time by forcing the user to be on pace with the sampler. The system works relatively accurately for monophonic inputs but falls short in certain aspects. The most common type of issue is an octave error, when multiple peaks remain after the HPS and the pitch detector cannot know for certain which peak to choose. This shows that pitch detection is easy to implement, but difficult to perfect. 

The product of this work is a JavaScript class and utility functions which could be integrated in another application. The project repository contains a simple frontend that can be used to run the pitch detector. The pitch detector is built in a modular way that would allow customization, even if there is no proper API available. Future development could involve restructuring the code so that a user can overload two functions, one for the analyzing function and the other for setting up the audio graph. As seen with the audio recordings from FSM, simply changing the input node is insufficient, the user may need to modify the audio graph as well. The application in its current state is also not minimized or optimized in any way.

The results are mixed. The pitch detection works and gives accurate results most of the time. The most common error is one where the detected note is an octave (or two octaves, in some cases) off. This seems to be a limitation of HPS and testing with different number of multiplication iterations reveals that many of the octave errors could be mitigated with a version of the HPS that could adjust the number of iterations depending on the number of peaks in the spectrum.
