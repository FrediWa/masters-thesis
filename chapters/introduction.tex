To tune a guitar, someone with perfect pitch perception can simply pluck the string while turning the corresponding tuning peg and stop when it's right. A person without perfect pitch would need to use a tuning fork or another sound source they know to be correct and compare the two until the difference is negligible. Both these processes are examples of pitch detection. With the evolution of computers, this processes has been improved as well with both dedicated devices for tuning guitars and smartphone applications. How a computer can detect pitch has been explored in depth and has yielded many different approaches. The pitch of a sound, how high or low it sounds, is tied to how fast a signal oscillates. Some methods look at the signal directly while others transform the signal in some way which makes certain aspects of pitch detection easier to conceptualize.

\subsection{Motivator for the topic}
For my bachelor's thesis, I created an web application which displays sheet music, can play the notes of the sheet music (with limitations) and pitch recognition for grading the singer's ability to hit notes. Most of application worked very well and all the things that I set out to implement worked well, except for the pitch detection. I used a neural network-based model called CREPE that ran as a part of a JavaScript library called ml5.js. I ended up with this approach because, from testing various pitch detectors, this one seemed to be the best working. It still was not very accurate and would often give back the wrong note which hurt the user's ability to learn correctness. The work focused on the entirety of the applications and the scope combined with the lack of expertise on my part made me not explore the pitch detection part further.  

\subsection{Problem statement}
The purpose of this work is to explore how the fast Fourier transform may be used for real-time pitch detection and what additional techniques are required to get reliable pitch detection. The algorithm should be able to detect pitches from a microphone stream in more or less real-time and should output precise notes. It should ideally be used in a web application, as this makes the application available on a wide range of devices but can also be used on-the-go, for example, for choir entrance tests or joint practice. The goal of this work is a reliable pitch detector that can be integrated in an existing application. The end user of that application should be able to trust the result of the pitch detection. 
