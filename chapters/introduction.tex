For my bachelor's thesis, I created an web application which displays sheet music, can play the notes of the sheet music (with limitations) and pitch recognition for grading the singer's ability to hit notes. Most of application worked very well and all the things that I set out to implement worked well, except for the pitch detection. I used a neural network-based model called CREPE that ran as a part of a JavaScript library called ml5.js. I ended up with this approach because, from testing various pitch detectors, this one seemed to be the best working. It still was not very accurate and would often give back the wrong note which hurt the user's ability to learn correctness. The work focused on the entirety of the applications and the scope combined with the lack of expertise on my part made me not explore the pitch detection part further.  

The purpose of this work is to explore how the fast Fourier transform may be used for real-time pitch detection. The algorithm should be able to detect pitches from a microphone stream in more or less real-timeand it should ideally be used in a web application, as this makes the application available on a wide range of devices but can also be used on-the-go, for example, for choir entrance tests or joint practice. The platform was already created in my bachelors thesis, so ideally, the product of this thesis should plug-and-play in the existing code base.


